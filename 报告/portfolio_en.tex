\documentclass[a4paper,12pt]{article}
\usepackage[UTF8]{ctex}
\usepackage{geometry}
\usepackage{hyperref}

% Page geometry
\geometry{left=2.5cm,right=2.5cm,top=2.5cm,bottom=2.5cm}

% Title formatting
\title{\textbf{Project "Alive" Portfolio}}
\author{
    \textbf{Leader:} 陈艺彬 (25121360) \\
    \textbf{Members:} 索昊 (25121366) \\
    \hspace{1.3cm} 卢晓晴 (25126579) \\
    \hspace{1.3cm} 李艺攀 (25126566) \\
    \hspace{1.3cm} 钱波 (25121365) \\
    \hspace{1.3cm} 张至柔 (25126625)
}
\date{December 30, 2025}

% Custom environment for daily logs
\newenvironment{dailylog}[2]{
    \noindent\rule{\textwidth}{1pt}\\[0.5em]
    \noindent\textbf{#1} \hfill \textbf{Task:} #2 \\[0.5em]
}{
    \vspace{1em}
}

\begin{document}

\maketitle

\section*{Development Log}

\begin{dailylog}{December 23, 2025}{1) Form a team, 2) Identify a project}
    \textbf{Achievements:}
    \begin{enumerate}
        \item \textbf{Team formed. Members include:}
        \begin{itemize}
            \item 陈艺彬 - 25121360 (Group Leader)
            \item 索昊 - 25121366
            \item 卢晓晴 - 25126579
            \item 李艺攀 - 25126566
            \item 钱波 - 25121365
            \item 张至柔 - 25126625
        \end{itemize}
        \item \textbf{Project Title:} Alive
        \item \textbf{Project Description:} "Alive" is a psychological support software designed for university students and researchers experiencing "sub-health emotions" and "hidden anxiety." Unlike traditional medical consultation apps, "Alive" focuses on emotional resonance, social companionship, and lightweight psychological redemption. It aims to build a "contemporary college student spiritual self-help mutual aid community," helping users find low-threshold meanings of "living" amidst "involution" through features like "Daily Survival Confirmation" and the "Forgetfulness Tree Hole." \textbf{Technically, we plan to use Flutter for cross-platform development and SQLite for encrypted local data storage to implement a "Local-First" architecture, ensuring ultimate privacy security.}
    \end{enumerate}
    \textbf{Gaps between Task and Achievements:}
    \begin{itemize}
        \item The core concept is defined, but the specific technical architecture for the "Tree Hole" feature (the connection between the offline version and the online database) requires further group discussion.
    \end{itemize}
\end{dailylog}

\begin{dailylog}{December 24, 2025}{Identify Stakeholders}
    \textbf{Achievements:}
    \begin{enumerate}
        \item \textbf{Stakeholder 1: University/Graduate Students (Core Users).}
        \begin{itemize}
            \item Description: Populations facing intense academic competition, employment pressure, and social expectations, sitting at the center of "involution." They often do not consider themselves mentally ill but feel persistently exhausted. They need a de-professionalized space to speak the truth.
        \end{itemize}
        \item \textbf{Stakeholder 2: Product Development Team (Us).}
        \begin{itemize}
            \item Description: Responsible for product design and implementation. Key responsibilities include designing a "Dopamine" style UI, ensuring privacy security, and controlling the warm, "anti-involution, anti-anxiety" tone of the software.
        \end{itemize}
    \end{enumerate}
    \textbf{Gap:}
    \begin{itemize}
        \item We need to conduct preliminary user interviews to verify if the concept of "treating living as a check-in" genuinely resonates with the target group.
        \item \textbf{Conflict of Interest:} We identified a core conflict between "regulatory requirements for real-name traceability" and "user desire for complete anonymity." This is not just a technical issue but a product ethics one, requiring further exploration of \textbf{"De-identification"} technologies.
    \end{itemize}
\end{dailylog}

\begin{dailylog}{December 25, 2025}{Identify Stakeholders}
    \textbf{Achievements:}
    \begin{enumerate}
        \item \textbf{Stakeholder 3: Regulatory Bodies / Platform Auditors.}
        \begin{itemize}
            \item Description: Entities responsible for enforcing internet content regulations. Since mental health is a sensitive topic, the software must have built-in "desensitization processing" and psychological warning mechanisms to avoid crossing regulatory red lines.
        \end{itemize}
        \item \textbf{Stakeholder 4: University Administration / Student Counseling Centers.}
        \begin{itemize}
            \item Description: Potential partners for promotion, but also potential sources of resistance. They prioritize student safety and might require the software to have intervention protocols for high-risk users.
        \end{itemize}
        \item \textbf{Stakeholder 5: Content Creators (Marketing/Traffic).}
        \begin{itemize}
            \item Description: Bloggers on Xiaohongshu/Weibo, responsible for creating impactful "resonance literature" posters to drive traffic to the App.
        \end{itemize}
    \end{enumerate}
    \textbf{Gap:}
    \begin{itemize}
        \item We have not yet contacted professional psychological counselors or volunteers, so we cannot verify if our "psychological warning mechanism" is sufficiently robust from a professional perspective.
    \end{itemize}
\end{dailylog}

\begin{dailylog}{December 26, 2025}{Identify Requirements}
    \textbf{Achievements:}
    \begin{enumerate}
        \item \textbf{Requirement Type 1: Functional (Core Gameplay). Name: Daily Survival Confirmation. Source: Core Users.}
        \begin{itemize}
            \item \textit{Details:} A low-threshold "check-in" system. Users simply confirm "I am alive today" as the day's metric. Aims to alleviate academic and employment anxiety, providing a "memory point storage" for the day.
        \end{itemize}
        \item \textbf{Requirement Type 2: Functional (Gamification). Name: Life Visualization / Emotional Vegetation. Source: Core Users / Team Idea.}
        \begin{itemize}
            \item \textit{Details:} Visualizing "Survival Confirmation." Every time a user confirms "I am alive," a virtual plant/pet grows. \textbf{Core Design Principle: Only provide positive feedback; long-term absence only leads to "dormancy" rather than "death" to avoid creating new anxiety.}
        \end{itemize}
    \end{enumerate}
    \textbf{Gap:}
    \begin{itemize}
        \item The algorithm logic for "Online Mode" and its content moderation costs are high, so it may be postponed to the second phase of development.
    \end{itemize}
\end{dailylog}

\begin{dailylog}{December 27, 2025}{Identify Requirements}
    \textbf{Achievements:}
    \begin{enumerate}
        \item \textbf{Requirement Type 3: Non-Functional (UI/UX). Name: Adaptive Emotional UI. Source: Dev Team / Human-Centered Design.}
        \begin{itemize}
            \item \textit{Details:} While focusing on \textbf{"Dopamine-inducing Color Psychology,"} we must provide a \textbf{"Night / Low Energy Mode."} For anxious users late at night, we reduce color saturation and blue light to avoid visual stimulation aggravating insomnia or mood swings.
        \end{itemize}
        \item \textbf{Requirement Type 4: Non-Functional (Safety/Compliance). Name: Anonymity and Crisis Intervention. Source: Regulators \& Users.}
        \begin{itemize}
            \item \textit{Details:} Must guarantee complete user anonymity to encourage truth-telling; simultaneously must embed automatic keyword detection to immediately trigger warning mechanisms upon identifying self-harm/suicide tendencies.
        \end{itemize}
    \end{enumerate}
    \textbf{Gap:}
    \begin{itemize}
        \item Balancing "guaranteeing complete anonymity (no real-name collection)" and "effective intervention in life-threatening situations" presents a technical difficulty that needs to be resolved.
    \end{itemize}
\end{dailylog}

\begin{dailylog}{December 28, 2025}{Refine Requirements}
    \textbf{Achievements:}
    \begin{enumerate}
        \item \textbf{Refine Requirement 1: Daily Survival Confirmation. Description: Is it measurable? Yes.}
        \begin{itemize}
            \item \textit{Metric:} The system records the count and timestamps of users clicking "Confirm Alive." The success standard is the percentage of Daily Active Users (DAU) using this feature.
        \end{itemize}
        \item \textbf{Refine Requirement 2: Crisis Warning Mechanism. Description: Is it measurable? Yes.}
        \begin{itemize}
            \item \textit{Metric:} The system must identify preset "high-risk keywords" within 1 second of user input, and ensure 100\% pop-up of a window containing local help hotlines.
        \end{itemize}
    \end{enumerate}
    \textbf{Gap:}
    \begin{itemize}
        \item We need to define the specific "high-risk keyword library" for Version 1 to ensure no false positives (mistaking venting for suicide risk) and no false negatives.
        \item \textbf{Risk of Community Entropy:} Pure venting can cause the community atmosphere to deteriorate rapidly (becoming an \textbf{"echo chamber of negativity"}). We are considering introducing a \textbf{"Positive Feedback Algorithm,"} where users must offer encouragement to others to unlock more Tree Hole content, thereby maintaining community warmth.
    \end{itemize}
\end{dailylog}

\begin{dailylog}{December 29, 2025}{Success Metrics}
    \textbf{Achievements:}
    \textbf{Description of Success Metrics:}
    \begin{enumerate}
        \item \textbf{User Retention Rate:} As an App focused on "companionship," the goal is a 7-day retention rate of > 30\%.
        \item \textbf{Emotional Relief Score (Qualitative):} Via an in-app brief survey: "Did you feel less anxious after checking in?" Target is to obtain > 70\% affirmative responses.
        \item \textbf{Safety Compliance:} Zero missed high-risk warnings; Zero leakages of real user identity data.
        \item \textbf{False Positive Rate:} The crisis warning system's false positive rate must be kept under 5\% to avoid annoying users and causing uninstalls.
        \item \textbf{Community Content Volume:} The number of valid "wish list" items accumulated in the Tree Hole database.
    \end{enumerate}
\end{dailylog}

\begin{dailylog}{December 30, 2025}{PPT Creation}
    \textbf{Achievements:}
    \begin{enumerate}
        \item \textbf{Completed Project Presentation (PPT), covering:}
        \begin{itemize}
            \item \textbf{Pain Point Analysis:} Students' hidden anxiety and involution status.
            \item \textbf{Solution:} Alive App (Survival Check-in + Tree Hole).
            \item \textbf{Stakeholder Analysis:} How to balance student needs with regulatory requirements.
            \item \textbf{Prototype Demo:} Showcasing Dopamine UI and anonymous interaction flow.
            \item \textbf{Future Roadmap:} Phase 1 (Offline Tool) $\rightarrow$ Phase 2 (Mutual Aid Community). Defined the \textbf{Development Roadmap}, clarifying the MVP (Minimum Viable Product) plan for the winter break.
        \end{itemize}
    \end{enumerate}
    \textbf{Gap:}
    \begin{itemize}
        \item None. PPT is ready for review.
    \end{itemize}
\end{dailylog}

\end{document}
