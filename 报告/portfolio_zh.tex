\documentclass[a4paper,12pt]{article}
\usepackage[UTF8]{ctex}
\usepackage{geometry}
\usepackage{hyperref}

% Page geometry
\geometry{left=2.5cm,right=2.5cm,top=2.5cm,bottom=2.5cm}

% Title formatting
\title{\textbf{Project "Alive" Portfolio}}
\author{
    \textbf{组长:} 25121360 陈艺彬 \\
    \textbf{组员:} 25121366 索昊 \\
    \hspace{1.3cm} 25126579 卢晓晴 \\
    \hspace{1.3cm} 25126566 李艺攀 \\
    \hspace{1.3cm} 25121365 钱波 \\
    \hspace{1.3cm} 25126625 张至柔
}
\date{2025年12月30日}

% Custom environment for daily logs
\newenvironment{dailylog}[2]{
    \noindent\rule{\textwidth}{1pt}\\[0.5em]
    \noindent\textbf{#1} \hfill \textbf{任务:} #2 \\[0.5em]
}{
    \vspace{1em}
}

\begin{document}

\maketitle

\section*{项目开发日志 (Daily Log)}

\begin{dailylog}{2025年12月23日}{1) 组建小组,2) 确定项目}
    \textbf{成果 (Achievements):}
    \begin{enumerate}
        \item \textbf{小组已组建。成员包括:}
        \begin{itemize}
            \item 陈艺彬 - 25121360 (组长)
            \item 索昊 - 25121366
            \item 卢晓晴 - 25126579
            \item 李艺攀 - 25126566
            \item 钱波 - 25121365
            \item 张至柔 - 25126625
        \end{itemize}
        \item \textbf{项目标题:} Alive
        \item \textbf{项目简述:} “Alive” 是一款专为处于“情绪亚健康”和“隐性焦虑”的大学生及研究人员设计的心理支持软件。不同于传统的医疗问诊类App,Alive 侧重于情绪共鸣、社交陪伴和轻量化的心理救赎。它旨在打造一个“当代大学生精神自救互助社群”,通过“每日生存确认”和“遗忘清单树洞”等功能,帮助用户在内卷中找到“活着”的低门槛意义。\textbf{技术上,拟采用“Flutter”进行跨平台开发,并使用“SQLite”进行本地数据加密存储以实现“本地优先(Local-First)”架构,确保极致的隐私安全。}
    \end{enumerate}
    \textbf{任务与成果之间的差距 (Gaps):}
    \begin{itemize}
        \item 核心概念已确定,但关于“树洞”功能的具体技术架构(离线版与在线版数据库的衔接)仍需小组进一步讨论。
    \end{itemize}
\end{dailylog}

\begin{dailylog}{2025年12月24日}{识别利益相关者 (Stakeholders)}
    \textbf{成果 (Achievements):}
    \begin{enumerate}
        \item \textbf{利益相关者 1:大学生/研究生(核心用户)。}
        \begin{itemize}
            \item 描述:面临激烈的学术竞争、就业压力以及社会期望,处于“内卷”中心的人群。他们往往不认为自己有精神疾病,但感到持续疲惫。他们需要一个去专业化、能说真话的宣泄空间。
        \end{itemize}
        \item \textbf{利益相关者 2:产品开发团队(我们)。}
        \begin{itemize}
            \item 描述:负责产品的设计与落地。关键职责包括设计“多巴胺”风格的UI,确保隐私安全,并把控软件“反内卷、反焦虑”的温情基调。
        \end{itemize}
    \end{enumerate}
    \textbf{Gap (差距):}
    \begin{itemize}
        \item 需要进行初步的用户访谈,以验证“把活着当成一种打卡”这一概念是否真能引起目标群体的共鸣。
        \item \textbf{利益冲突(Conflict of Interest):} 我们在“监管方要求的实名追溯”与“用户渴望的完全匿名”之间发现核心冲突。这不仅是技术问题,更是产品伦理问题,需要进一步探讨“去标识化(De-identification)”技术的可行性。
    \end{itemize}
\end{dailylog}

\begin{dailylog}{2025年12月25日}{识别利益相关者 (Stakeholders)}
    \textbf{成果 (Achievements):}
    \begin{enumerate}
        \item \textbf{利益相关者 3:监管机构 / 平台审核方。}
        \begin{itemize}
            \item 描述:负责执行网络内容法规的实体。由于心理健康属于敏感话题,软件必须内置完善的“脱敏处理”和心理预警机制,以避免触碰监管红线。
        \end{itemize}
        \item \textbf{利益相关者 4:高校管理层 / 学生心理咨询中心。}
        \begin{itemize}
            \item 描述:潜在的推广合作对象,也可能是阻力方。他们关注学生的安全,可能会要求软件具备针对高危用户的干预协议。
        \end{itemize}
        \item \textbf{利益相关者 5:内容创作者(营销引流)。}
        \begin{itemize}
            \item 描述:小红书/微博上的博主,负责制作具有冲击力的“共鸣文学”海报,为App引流。
        \end{itemize}
    \end{enumerate}
    \textbf{Gap (差距):}
    \begin{itemize}
        \item 目前尚未联系到专业的心理咨询师或志愿者,无法验证我们的“心理预警机制”在专业角度是否足够完善。
    \end{itemize}
\end{dailylog}

\begin{dailylog}{2025年12月26日}{识别需求 (Requirements)}
    \textbf{成果 (Achievements):}
    \begin{enumerate}
        \item \textbf{需求类型 1:功能性需求(核心玩法)。名称:每日生存确认。来源:核心用户。}
        \begin{itemize}
            \item \textit{细节:} 一个低门槛的“打卡”系统。用户只需确认“今天我活着”,作为当日的指标。旨在缓解学业和就业焦虑,提供当日的“记忆点存储”。
        \end{itemize}
        \item \textbf{需求类型 2:功能性需求(游戏化)。名称:生命可视化/情绪植被。来源:核心用户/团队创意。}
        \begin{itemize}
            \item \textit{细节:} 将“生存确认”具象化。用户每确认一次“活着”,虚拟界面中的植物/宠物获得一次成长。\textbf{核心设计原则是:只提供正向反馈,长期不打卡仅会“休眠”而非“死亡”,避免制造新的焦虑。}
        \end{itemize}
    \end{enumerate}
    \textbf{Gap (差距):}
    \begin{itemize}
        \item 关于“在线模式”的算法逻辑及其内容审核成本较高,可能暂时推迟到软件的第二阶段开发。
    \end{itemize}
\end{dailylog}

\begin{dailylog}{2025年12月27日}{识别需求 (Requirements)}
    \textbf{成果 (Achievements):}
    \begin{enumerate}
        \item \textbf{需求类型 3:非功能性需求(UI/UX)。名称:自适应情绪UI。来源:开发团队/人本设计。}
        \begin{itemize}
            \item \textit{细节:} 虽然主打“多巴胺配色”,但需提供\textbf{“深夜/低能量模式”}。针对深夜使用的焦虑人群,降低色彩饱和度,减少蓝光,避免视觉刺激造成失眠加重或情绪波动。
        \end{itemize}
        \item \textbf{需求类型 4:非功能性需求(安全/合规)。名称:匿名性与危机干预。来源:监管机构与用户。}
        \begin{itemize}
            \item \textit{细节:} 必须保证用户的完全匿名性以鼓励说真话;同时必须植入自动关键词检测,一旦识别到自残/自杀倾向,立即触发预警机制。
        \end{itemize}
    \end{enumerate}
    \textbf{Gap (差距):}
    \begin{itemize}
        \item 如何在“保证完全匿名(不收集实名信息)”和“遇到生命危险时能有效干预”之间取得平衡,在技术实现上存在难点。
    \end{itemize}
\end{dailylog}

\begin{dailylog}{2025年12月28日}{完善需求 (Refine Requirements)}
    \textbf{成果 (Achievements):}
    \begin{enumerate}
        \item \textbf{完善需求 1:每日生存确认。描述:它可衡量吗?是的。}
        \begin{itemize}
            \item \textit{衡量标准:} 系统记录用户每日点击“确认活着”的次数和时间戳。成功标准是该功能的日活跃用户数(DAU)占比。
        \end{itemize}
        \item \textbf{完善需求 2:危机预警机制。描述:它可衡量吗?是的。}
        \begin{itemize}
            \item \textit{衡量标准:} 系统需在用户输入内容后的 1秒内 识别出预设的“高危关键词”,并确保 100\% 弹出包含本地求助热线的弹窗。
        \end{itemize}
    \end{enumerate}
    \textbf{Gap (差距):}
    \begin{itemize}
        \item 需要定义第一版具体的“高危关键词库”,以确保既不误报(把吐槽当自杀),也不漏报。
        \item \textbf{社区氛围熵增风险(Risk of Community Entropy):} 纯粹的宣泄会导致社区氛围迅速恶化(变为情绪垃圾桶)。我们正在考虑引入“正向反馈算法”,即只有在用户给予他人鼓励后,才能解锁更多树洞内容,以此维持社区的温暖度。
    \end{itemize}
\end{dailylog}

\begin{dailylog}{2025年12月29日}{成功指标 (Success Metrics)}
    \textbf{成果 (Achievements):}
    \textbf{成功指标描述:}
    \begin{enumerate}
        \item \textbf{用户留存率 (Retention Rate):} 作为一个主打“陪伴”的App,目标是7日留存率达到 30\% 以上。
        \item \textbf{情绪缓解评分 (Emotional Relief Score):} (定性指标) 通过应用内简短调研:“打卡后你是否感觉焦虑减轻?”,目标是获得 70\% 的肯定回答。
        \item \textbf{安全合规性 (Safety Compliance):} 高危预警零漏报;用户真实身份数据零泄露。
        \item \textbf{误报率 (False Positive Rate):} 危机预警系统的误报率需控制在 5\% 以下,以免频繁骚扰用户导致卸载。
        \item \textbf{社区内容量:} 树洞数据库中积累的有效“愿望清单”数量。
    \end{enumerate}
\end{dailylog}

\begin{dailylog}{2025年12月30日}{PPT 制作}
    \textbf{成果 (Achievements):}
    \begin{enumerate}
        \item \textbf{完成了项目演示文稿 (PPT),内容涵盖:}
        \begin{itemize}
            \item \textbf{痛点分析:} 学生的隐性焦虑与内卷现状。
            \item \textbf{解决方案:} Alive App (生存打卡 + 树洞)。
            \item \textbf{利益相关者分析:} 如何平衡学生需求与监管要求。
            \item \textbf{原型演示:} 展示多巴胺UI配色与匿名交互流程。
            \item \textbf{未来规划:} 第一阶段(离线工具) $\rightarrow$ 第二阶段(互助社区)。制定了下一步开发路线图(Roadmap),明确了寒假期间的 MVP(最小可行性产品)开发计划。
        \end{itemize}
    \end{enumerate}
    \textbf{Gap (差距):}
    \begin{itemize}
        \item 无。PPT已准备好接受审核。
    \end{itemize}
\end{dailylog}

\end{document}
